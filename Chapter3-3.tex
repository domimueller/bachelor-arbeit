% ================================================================
% CHAPTER 3: Domänenanalyse: Domänenmodell
% ================================================================



\section{Domänenmodell}
\subsection{Legende zu den Diagrammen}
\begin{table}[H]
	\centering	
	\rowcolors{2}{maroon!10}{white!100}
	\arrayrulecolor{darkmaroon} 
	
	\begin{tabular}{ p{2cm}  p{12cm}  }
		
		\toprule[1pt]
		\rowcolor{maroon!30}	
		Farbe & Beschreibung \\
		
		\midrule
Grün \cellcolor[RGB]{204,235,197}&  Natürliche Gegenstände im Domänenmodell und Value Objects in der Bibliothek, welche diese natürlichen Gegenstände beschreiben. \\
Lila \cellcolor[RGB]{253,218,236} & Ressourcen und Faktoren, welche es zu optimieren gilt und Value Objects in der Bibliothek,  welche diese zu optimierenden Gegenstände beschreiben.\\
Blau\cellcolor[RGB]{179,205,227} &  Objekte oder Resultate der Domäne Image-Analysis. Zudem Value Objects in der Bibliothek, welche die Gegenstände dieser Domäne beschreiben. \\
Braun \cellcolor[RGB]{229,216,189}& Tatsächlicher Kern der Domäne Image-Analysis. Dieser Kern implementiert das wichtigste Domänenwissen und zentrale Funktionen.\\			
Violett \cellcolor[RGB]{222,203,228} & Objekte, die zum Versenden von Benachrichtigungen benötigt werden und Value Objects in der Bibliothek, welche die Gegenstände dieser Domäne beschreiben.\\
Grau\cellcolor[RGB]{242,242,242} &  Paket für die Konfiguration der Benachrichtigungen und Value Objects in der Bibliothek, welche die Gegenstände dieser Domäne beschreiben.\\		
Gelb \cellcolor[RGB]{255,255,204}& Value Objects, welche direkt im Domänenmodell oder in der Lösungsdokumentation eingebettet sind. Zudem Value Objects die nicht kategorisiert werden können, da diese in mehreren Paketen eingesetzt werden. \\		
Orange \cellcolor[RGB]{254,217,166} &  Paket Housekeeping und Klassen vom Paket ImageAnalysis, welche dem Housekeeping dienen. \\
		
		\bottomrule
		
	\end{tabular}
	\caption{Legende fürs Domänenmodell und für die UML-Diagramme als Lösungsdokumentation}
	\label{tab: Legende fürs Domänenmodell und für die UML-Diagramme als Lösungsdokumentation}
\end{table}

\newgeometry{margin=2.5cm} % Ränder kleiner	
\begin{landscape}

\subsection{Domänenmodell Geburt}
\begin{figure}[H]
	\center
	\includegraphics[scale=0.38]{Grafiken/modelle/domain-birth.jpg}
	\caption{Domänenmodell Kalbsgeburt} 
	\label{fig: Domänenmodell Kalbsgeburt}
\end{figure}


\subsection{Allgemeine Bibliothek an Value Objects }
\begin{figure}[H]
	\center
	\includegraphics[scale=0.4]{Grafiken/modelle/vo-general.jpg}
	\caption{Allgemeine Bibliothek an Value Objects} 
	\label{fig: Allgemeine Bibliothek an Value Objects}
\end{figure}

\subsection{Bibliothek an Value Objects zur Konfiguration und zum Versand von Nachrichten}
\begin{figure}[H]
	\center
	\includegraphics[scale=0.38]{Grafiken/modelle/vo-messaging.jpg}
	\caption{Bibliothek an Value Objects bezüglich Konfiguration und Versand von Nachrichten} 
	\label{fig: Bibliothek an Value Objects zur Konfiguration und zum Versand von Nachrichten}
\end{figure}


\end{landscape}
\restoregeometry % Wieder die alten Ränder