% ================================================================
% CHAPTER 4: Lösung
% ================================================================

	
\chapter{Lösung}

\section{Codierung von Domänenwissen}

Tabelle xx (<<VERWEIS AUF TABELLE EINFÜGEN>>) codiert Domänenwissen von Tierärzten, welches für die Geburtsprognose von Bedeutung ist. Expertenwissen ist so codiert, dass jedes Geburtsmerkmal mit einer Gewichtung versehen ist. Somit hat das Merkmal an Stelle $i$ die Gewichtung $\lambda_{i}$.



Wie in Formel \ref{Linerarkombination zur Geburtsprognose} formalisiert, werden bei Erkennung eines oder mehrerer Merkmale in einem Bild, die Gewichtungen dieser Merkmale addiert. Merkmale, welche als Geburtsanzeichen dienen, haben eine positive Gewichtung und erhöhen dadurch das Endergebnis. Merkmale, welche darauf hinweisen, dass zurzeit keine Entbindung stattfindet, haben eine negative Gewichtung und senken das Endergebnis.


Das Resultat dieser Berechnung bezeichnet der Autor als \flqq qualitative Situationsbewertung\frqq. Diese wird mit einem Schwellwert verglichen, um zu entscheiden, ob eine Benachrichtigung verschickt wird.

Dabei kann die qualitative Situationsbewertung anhand der folgenden Linearkombination ermittelt werden: 



\begin{equation}\label{Linerarkombination zur Geburtsprognose}
v = h(x) = \sum_{i=1}^n \lambda_{i}*x_{i}
\end{equation}

 
Wobei $\lambda_{i}$ sich aus der fachlichen Gewichtung $\kappa_{i}$ und der technischen Qualitätsbeurteilung $\beta_{i}$ zusammensetzt. Die Skala zur Bewertung reicht von -10 bis 10, daher ergibt sich $ \lambda = \kappa = \beta = \{ \:  m \: | \:  m \:  \varepsilon \:  \mathbb{Z}, -10 \:  \leq \: m \: \leq \:  10\} $.

Die fachliche Gewichtung eines Merkmals entspricht der Bewertung  der Stärke des Hinweises in Bezug auf eine bevorstehende Geburt. Dementsprechend codiert $\kappa$ Domänenwissen von Tierärzten zur merkmalsbezogene Einschätzung und Prognose des Geburtsverlaufs.

Die technische Qualitätsbeurteilung basiert auf der Güte der technischen Mittel zwecks Analyse der An- oder Abwesenheit eines Merkmals in einem Bild. Dementsprechend wird $\beta$ benötigt, weil nicht sämtliche Merkmale mit derselben Qualität auf deren An- oder Abwesenheit überprüft werden können.

Zwischen den beiden Gewichtungen $\kappa$ und $\beta$ existiert eine schwache Beziehung. Dies wird dadurch begründet, dass technische Zuverlässigkeit die Anwesenheit eines Merkmals nicht überbewerten soll. Beispielsweise soll eine besonders hohe technische Qualitätsbeurteilung und eine tiefe fachliche Gewichtung nicht in einer hohen Gewichtung resultieren. Um diese zwei Gewichtungen schwach miteinander in Beziehung zu setzen, werden diese addiert und nicht multipliziert. Es gilt also: $\lambda_{i} = \kappa_{i} + \beta_{i}$

Zudem codiert $x_{i}$ die Anwesenheit ($x_{i}=1$) oder die Abwesenheit ($x_{i}=0$) eines spezifischen Merkmals auf einem Bild. Es gilt dementsprechend  $ x_{i} \:  \varepsilon \: \{0,1\}. $ 


Setzen wir diese Erkenntnisse zusammen, so ergibt sich

\begin{equation}\label{Vollständige Linerarkombination zur Geburtsprognose}
v = h(x) = \sum_{i=1}^n (\kappa_{i}+\beta_{i}) *x_{i}
\end{equation}

wobei:  $ x \:  \varepsilon \: \{0,1\} $ und $\kappa = \beta = \{ \:  m \: | \:  m \:  \varepsilon \:  \mathbb{Z}, -10 \:  \leq \: m \: \leq \:  10\} $.

Das Ergebnis dieser Rechenoperation ergibt wie bereits erwähnt die qualitative Situationsbewertung $v$, welche mit einem Schwellwert verglichen wird. Dies wird formal wie folgt ausgedrückt:

\begin{equation}\label{Vollständige Linerarkombination zur Geburtsprognose}
y = g(v) =\begin{cases}
			1,\: wenn \: v > q\\
			0,\: wenn \: v \leq q
\end{cases}
\end{equation}

Dabei steht $q$ für den Schwellwert, welcher in der Domänenanalyse ermittelt wird. Das Ergebnis $y=1$ bedeutet, dass eine Benachrichtigung ausgelöst wird, während bei $y=0$ keine Benachrichtigung ausgelöst wird.

 






\newgeometry{margin=2.5cm} % Ränder kleiner	
\begin{landscape}

\section{Modellierung der Lösung}
Für die Dokumentation der Lösung verwendet der Autor die UML-Notation. Zwei Kassendiagramme beschreiben das System aus jeweils unterschiedlichen Perspektiven \cite{Uml-modellierung2012}. 

\subsection{Klassendiagramm des Pakets Image-Analysis }
\begin{figure}[H]
	\center
	\includegraphics[scale=0.45]{Grafiken/modelle/solution-imageanalysis.jpg}
	\caption{Lösungsdokumentation des Pakets Image-Analysis zur Geburtsprognose und Geburtserkennung.} 
	\label{fig: Lösungsdokumentation des Pakets Image-Analysis zur Geburtsprognose und Geburtserkennung.}
\end{figure}

\subsection{Klassendiagramm der Pakete Message-Configuration und Message-Posting}
\begin{figure}[H]
	\center
	\includegraphics[scale=0.43]{Grafiken/modelle/solution-messaging.jpg}
	\caption{Lösungsdokumentation der Pakete Message-Configuration zur Konfiguration und Message Posting zum Versand von Benachrichtigungen.} 
	\label{fig: Lösungsdokumentation der Pakete Message-Configuration zur Konfiguration und Message Posting zum Versand von Benachrichtigungen.}
\end{figure}

\subsection{Value-Object-Bibliothek}
Die Bibliothek an Value Objects wird gemäss Kapitel 3 Domänenanalyse eingesetzt.

Zusammenfassend ist erwähnenswert,... 

\end{landscape}
\restoregeometry % Wieder die alten Ränder

\section{Überlegungen für den Betrieb}

Um einen stabilen Betrieb von elektronischen Geräten und dem Netzwerk auf einem Bauernhof sicherzustellen, müssen folgende Überlegungen gemacht werden:


\newpage


\section{Umsetzung in Entwicklung}
Um die schrittweise Entwicklung und entsprechende Teilergebnisse der Bildbearbeitung zu veranschaulichen, dient die zufällig ausgewählte Abbildung \ref{fig: Bild als Ausgangslage zur Veranschaulichung des Vorgehens} als Ausgangsbild. Dieses Bild wurde vom System erstellt, welches im Rahmen der erwähnten Case-Arbeit entwickelt wurde. Anschliessend wurde das Bild mit der Funktion \texttt{add()} heller gemacht und mit den Funktionen \texttt{createCLAHE()} und \texttt{clahe.apply()} wurde das Histogramm des Bildes geglättet. 

\begin{figure}[H]
	\center
	\includegraphics[scale=0.43]{Grafiken/entwicklung/1ausgangsbildBericht.jpg}
	\caption{Beispielbild als Ausgangslage zur Veranschaulichung des Vorgehens} 
	\label{fig: Bild als Ausgangslage zur Veranschaulichung des Vorgehens}
\end{figure}

Abbildung \ref{fig: Vergleich der Histogramme vor und nach Bildbearbeitung} zeigt auf der linken Seite das Histogramm des Originalbilds und auf der rechten Seite das Histrogramm des aufgehellten und geglätteten Bilds. Dabei fällt auf, dass aufgrund der Aufhellung im Histogramm des Bearbeiteten Bilds mehr Werte im rechten Bereich liegen und aufgrund der Glättung die Anzahl Pixel weniger stark auf einen Bereich konzentriert sind.

\begin{figure}[H]
	\center
	\includegraphics[scale=0.9]{Grafiken/entwicklung/2HistrogrammVergleich.jpg}
	\caption{Vergleich der Histogramme vor und nach Bildbearbeitung} 
	\label{fig: Vergleich der Histogramme vor und nach Bildbearbeitung}
\end{figure}

\subsection{Detektierung von unwichtigen Bereichen im Bild}
In einem ersten Schritt setzt sich der Autor zum Ziel, unwichtige Bereiche im Bild zu identifizieren. Dies umfasst Bereiche, welche mit hoher Wahrscheinlichkeit nicht Teile einer Kuh oder eines Kalbs zeigen. Im betrachteten Bild wird zuerst versucht, die Lampe, den schwach beleuchteten Teil des Stallbodens und den Holzträger zu erkennen. Die Farbwerte dieser Teile im Bild unterscheiden sich stark zu den meisten Farben im Kuhfell.

Um dies zu erreichen, wird in einem ersten Schritt ein Farbbereich für die Lampe definiert. Um einen Richtwert für diesen Farbwert zu erhalten, wird mithilfe des Grafikprogramms GIMP (Version 2.10, \url{www.gimp.org}) ein Farbwert im Bereich der Lampe ausgelesen. Auf Basis dieses Richtwerts kann anschliessend ein unterer und oberer Schwellwert für die Farbwerte der zu identifizierenden Lampe definiert werden. Diese Schwellwerte werden der Funktion \texttt{inRange()} von OpenCV zwecks Erstellung eines Binärbilds übergeben. Sämtliche Bereiche mit Farbwerten, die sich zwischen den definierten Schwellwerten befinden, werden im resultierenden Bild weiss dargestellt. Diese entsprechen den vermutlich unwichtigen Bereichen im Bild. Alle anderen Bereiche sind im erstellen Binärbild schwarz dargestellt \ref{fig: Binärbild, welches den Bereich des Stallbodens und Holzträgers weiss darstellt}. Dieses Binärbild identifiziert auch einige dunkle Regionen im Deckenbereich als unwichtig.

\begin{figure}[H]
	\center
	\includegraphics[scale=0.43]{Grafiken/entwicklung/3binBildLampe.jpg}
	\caption{Binärbild, welches den Bereich der Lampe weiss darstellt} 
	\label{fig: Binärbild, welches den Bereich der Lampe weiss darstellt}
\end{figure}

Dasselbe Vorgehen wird angewendet, um den Bereich des Stallbodens und Holzträgers zu identifizieren.

\begin{figure}[H]
	\center
	\includegraphics[scale=0.43]{Grafiken/entwicklung/4binBildHolz.jpg}
	\caption{Binärbild, welches den Bereich des Stallbodens und Holzträgers weiss darstellt} 
	\label{fig: Binärbild, welches den Bereich des Stallbodens und Holzträgers weiss darstellt}
\end{figure}

Aus diesen zwei Binärbildern wird nun ein Binärbild erstellt, welches sämtliche Regionen weiss darstellt, die in einem oder beiden Bildern bereits weiss sind. Dies bedeutet, dass die in einem oder in beiden Binärbilden als unwichtig identifizierten Bereiche in einem neuen Binärbild kombiniert werden. Um dies zu erreichen, wird die Funktion \texttt{bitwise_or} von OpenCV mit den beiden Binärbildern als Argumente aufgerufen. 

\begin{figure}[H]
	\center
	\includegraphics[scale=0.43]{Grafiken/entwicklung/5binLampeUndHolz.jpg}
	\caption{Binärbild, welches die vorherigen Binärbilder von unwichtigen Regionen kombiniert} 
	\label{fig: Binärbild, die vorherigen Binärbilder kombiniert}
\end{figure}

In einem weiteren Schritt können die als unwichtig identifizierten Bereiche als Konturen erkannt und somit ins Ausgangsbild eingezeichnet werden. Um dies zu erreichen, werden die Funktionen \texttt{findContours()} und \texttt{drawContours()} angewendet. Das Resultat ist in Abbildung \ref{fig: Unwichtige Bereiche als Konturen} veranschaulicht.
\begin{figure}[H]
	\center
	\includegraphics[scale=0.43]{Grafiken/entwicklung/6unwichtigeBereicheEingezeichnet.jpg}
	\caption{Unwichtige Bereiche  als Konturen} 
	\label{fig: Unwichtige Bereiche als Konturen}
\end{figure}

Nun gilt es, anhand von diesen als unwichtig identifizierten Konturen möglichst grosse Flächen aus dem Ausgangsbild zu löschen, respektive möglichst grosse Flächen mit schwarzer Farbe zu füllen. Um dies zu erreichen, wurden mehrere Verfahren getestet. Die Abbildungen  \ref{fig: Unwichtige Bereiche als Polygone} bis \ref{fig: Unwichtige Bereiche als Rechtecke} veranschaulichen die entsprechenden Resultate unter Verwendung von OpenCV. In Abbildung \ref{fig: Unwichtige Bereiche als Polygone} wurde die Funktion \texttt{approxPolyDP()} verwendet, um basierend auf den Konturen ein Vieleck zu approximieren. Abbildung \ref{fig: Unwichtige Bereiche als konvexe Hülle} zeigt die Ergebnisse der Funktion \texttt{convexHull()}. Die Abbildung \ref{fig: Unwichtige Bereiche als Kreise} nutzt die Funktion \texttt{minEnclosingCircle()} um Kreise zu finden, welche Konturen mit möglichst geringer Fläche umschliessen. Zuletzt wurden für die Abbildung \ref{fig: Unwichtige Bereiche als Rechtecke} mit der Funktion \texttt{boundingRect()} Rechtecke gebildet, welche die Konturen umschliessen .

Als Vorbereitung für die Durchführung von diesen Verfahren wurde mittels \texttt{contourArea()} jeweils die Fläche der Kontur berechnet und nur berücksichtigt, wenn diese mehr als 200 Pixel umfasst. Dadurch wird erreicht, dass beispielsweise Schatten unterhalb des Schwanzes oder der unterhalb der Beine der Kuh nicht berücksichtigt werden.
\begin{figure}[H]
	\center
	\includegraphics[scale=0.43]{Grafiken/entwicklung/7unwichtigePolygone.jpg}
	\caption{Unwichtige Bereiche als Polygone } 
	\label{fig: Unwichtige Bereiche als Polygone}
\end{figure}


\begin{figure}[H]
	\center
	\includegraphics[scale=0.43]{Grafiken/entwicklung/7unwichtigeKonvexe.jpg}
	\caption{Unwichtige Bereiche als konvexe Hülle} 
	\label{fig: Unwichtige Bereiche als konvexe Hülle}
\end{figure}


\begin{figure}[H]
	\center
	\includegraphics[scale=0.43]{Grafiken/entwicklung/7unwichtigeKreise.jpg}
	\caption{Unwichtige Bereiche als Kreise } 
	\label{fig: Unwichtige Bereiche als Kreise}
\end{figure}

\begin{figure}[H]
	\center
	\includegraphics[scale=0.43]{Grafiken/entwicklung/7unwichtigeRechtecke.jpg}
	\caption{Unwichtige Bereiche als Rechtecke} 
	\label{fig: Unwichtige Bereiche als Rechtecke}
\end{figure}

Die Verfahren, welche mittels \texttt{minEnclosingCircle()} und \texttt{boundingRect()} angewendet werden, ergeben im vorliegenden Kontext die besten Ergebnisse. Die als unwichtig identifizierte Flächen werden so maximiert und die Fehlerrate (Flächen von Kuh als unwichtig identifiziert) ist gering. Da perfekte Kreise in den Konturen von Kühen und Kälbern nicht vorkommen, entscheided sich der Autor dafür, unwichtige Bereiche als schwarze Kreise einzuzeichnen. Zudem können diese Kreise in der weiteren Bildbearbeitung problemlos als solche erkannt werden.

\subsection{Detektierung von wichtigen Bereichen im Bild}

Um nun das Potential aus der Detektierung von unwichtigen Bereichen zu veranschaulichen, wird in einem ersten Schritt ein Binärbild des Ausgangsbilds erstellt. Dazu wird die Funktion \texttt{threshold()} mit dem Verfahren \texttt{THRESH_BINARY} und der Zahl \texttt{40} als Schwellwert eingesetzt.

\begin{figure}[H]
	\center
	\includegraphics[scale=0.43]{Grafiken/entwicklung/8thresholdedMask.jpg}
	\caption{Binärbild, welches aus dem Ausgangsbild erstellt wurde} 
	\label{fig: Binärbild, welches aus dem Ausgangsbild erstellt wurde}
\end{figure}

In diesem Binärbild werden anschliessend analog zum Vorgehen bei der Detektierung von unwichtigen Bereichen unter Anwendung der Funktionen \texttt{findContours()} und \texttt{drawContours()} Konturen gesucht und im Ausgangsbild eingezeichnet.


\begin{figure}[H]
	\center
	\includegraphics[scale=0.43]{Grafiken/entwicklung/9thresholdedImage.jpg}
	\caption{Ausgangsbild mit rot eingefärbten Konturen} 
	\label{fig: Ausgangsbild mit rot eingefärbten Konturen} 
\end{figure}

Nun ist klar zu erkennen, dass auch Konturen eingezeichnet werden, die klar als unwichtige Bereiche identifizierbar sind (in erster Linie die Lampe).

Aus diesem Grund nutzt der Autor die Information über unwichtige Bereiche, um nur noch Konturen zu berücksichtigen, die nicht in diesem Bereich liegen. Dementsprechend wird das in Abbildung \ref{label{fig: Unwichtige Bereiche als Kreise}} dargestellte Bild zur weiteren Analyse verwendet. In einem ersten Schritt wird die geglättete und aufgehellte Version dieses Bilds mit dem adaptiven Schwellwertverfahren bearbeitet. Um dies zu erreichen, wird die Funktion \texttt{adaptiveThreshold()} mit den Argumenten \texttt{THRESH_BINARY_INV} und \texttt{ADAPTIVE_THRESH_MEAN_C} aufgerufen. Das Ergebnis dieses Versuchs, wichtige Bereiche des Bilds zu detektieren, ist in Abbildung \ref{{fig: Versuch, aus geglättetem und aufgehelltem Bild wichtige Bereiche zu detektieren}} ersichtlich. 

\begin{figure}[H]
	\center
	\includegraphics[scale=0.43]{Grafiken/entwicklung/10thresholdedEqualizedBirghtened.jpg}
	\caption{Versuch, aus geglättetem und aufgehelltem Bild wichtige Bereiche zu detektieren} 
	\label{fig: Versuch, aus geglättetem und aufgehelltem Bild wichtige Bereiche zu detektieren} 
\end{figure}

Das Ergebnis ist nicht brauchbar, weshalb der Versuch mit denselben Einstellungen aber mit einem nicht geglätteten Bild wiederholt wurde. Das Ergebnis ist in Abbildung \ref{{fig: Versuch, aus nicht geglättetem Bild wichtige Bereiche zu detektieren}} dargestellt und stellt ein besseres Zwischenergebnis dar. Als entsprechende Erkenntnis leitet der Autor ab, dass die Glättung von Histogrammen zwar die Bildqualität und Interpretationsfähigkeit für den Menschen steigert, aber im vorliegenden Kontext auch negativen Einfluss auf das Schwellwertverfahren haben kann.

\begin{figure}[H]
	\center
	\includegraphics[scale=0.43]{Grafiken/entwicklung/11thresholdedNotEqualized.jpg}
	\caption{Versuch, aus nicht geglättetem Bild wichtige Bereiche zu detektieren} 
	\label{fig: Versuch, aus nicht geglättetem Bild wichtige Bereiche zu detektieren} 
\end{figure}

Abbildung \ref{{fig: Versuch, aus nicht geglättetem Bild wichtige Bereiche zu detektieren}} zeigt das Resultat einer vielversprechenden Analyse. Das Ergebnis ist aber insofern kritisch zu beurteilen, da die rote Farbe die gesamte Kontur ausfüllt, die detektiert wurde. Dementsprechend werden die Beine beispielsweise nicht als Kontur erkannt, sondern lediglich die Umrisse davon.
Dies dient als Motivation, die Konfiguration des Schwellwertverfahren anzupassen. Demzufolge wurde mit der Funktion \texttt{threshold()} und den Argumenten \texttt{THRESH_BINARY} als Verfahrenstyp, und dem Wert \texttt{90} als Schwellwert. Das Ergebnis daraus ist in Abbildung \ref{{fig: Ergebnisse nach angepasster Konfiguration des Schwellwertverfahrens}} dargestellt. Zudem verbessert die Filterung von kleinen Konturen die Ergebnisse weiter. Dadurch werden die als unwichtig Identifizierten Kreise und Rausch wie beispielsweise Strohhaufen neben der Kuh nicht mehr berücksichtigt.
\begin{figure}[H]
	\center
	\includegraphics[scale=0.43]{Grafiken/entwicklung/12SimpleThresholdingConoturOutlineCCOMP.jpg}
	\caption{Ergebnisse nach angepasster Konfiguration des Schwellwertverfahrens} 
	\label{fig: Ergebnisse nach angepasster Konfiguration des Schwellwertverfahrens} 
\end{figure}
Es fällt nun auf, dass Flecken innerhalb der Kontur der Kuh erkannt werden. Da \texttt{findContours()} mit dem Argument \texttt{RETR_CCOMP} aufgerufen wird, werden auch Konturen innerhalb von den äusseren Konturen zurückgegeben und in eine  Hierarchie eingeteilt. Die äusseren Konturen entsprechen in den meisten Fällen den Umrissen der Kuh. Zum aktuellen Zeitpunkt reicht es aus, nur diese Umrisse zu erkennen und dementsprechend wird \texttt{findContours()} mit dem Argument \texttt{RETR_EXTERNAL} aufgerufen. Das Ergebnis ist in Abbildung  \ref{{fig: Ergebnisse nach angepasster Konfiguration des des Contour Finders}} sichtbar.
\begin{figure}[H]
	\center
	\includegraphics[scale=0.43]{Grafiken/entwicklung/13SimpleThresholdingConoturOutlineLIST.jpg}
	\caption{Ergebnisse nach angepasster Konfiguration des des Contour Finders} 
	\label{fig: Ergebnisse nach angepasster Konfiguration des des Contour Finders} 
\end{figure}

Dabei unterscheiden sich die Ergebnisse bei der Anwendung von \texttt{findContours()} mit unterschiedlichen Approximationsverfahren wie \texttt{CHAIN_APPROX_SIMPLE}, \texttt{CHAIN_APPROX_TC89_L1}, \texttt{CHAIN_APPROX_TC89_KCOS} von der Deaktivierung der Approximation  nicht wesentlich. Auch unterschiedliche Modi wie \texttt{RETR_LIST} oder \texttt{RETR_CCOMP} für die Identifizierung der Konturen ergeben in diesem Kontext identische Ergebnisse. Es werden jeweils die neun eingezeichneten Konturen identifiziert.

Da nun lediglich äussere Konturen detektiert werden, dürfen diese Konturen mit Farbe gefüllt werden, ohne relevante Informationen über Hierarchien zu vernichten (Abbildung \ref{fig: Konturen mit Farbe gefüllt}). 

\begin{figure}[H]
	\center
	\includegraphics[scale=0.43]{Grafiken/entwicklung/14AfterThresholdingContourFilled.jpg}
	\caption{Konturen mit Farbe gefüllt} 
	\label{fig: Konturen mit Farbe gefüllt} 
\end{figure}


\subsection{Erkennung von seitlich liegender Kuh}
Aus der Domänenanalyse und Experteninterviews hat sicher ergeben, dass seitliches Liegen ein starkes Geburtsanzeichen ist und entsprechend detektiert werden muss. Die Seitenlage charakterisiert sich unter anderem mit gestreckten Beinen. Um diese zu erkennen, 
Um die detektierten Konturen weiter einzugrenzen, werden diese nach entsprechenden Winkeln gefiltert. 


\begin{figure}[H]
	\center
	\includegraphics[scale=1.8]{Grafiken/entwicklung/21AngleCorrecturDemonstration.jpg}
	\caption{Veranschaulichung zur Analyse der Winkel} 
	\label{fig: Veranschaulichung zur Analyse der Winkel} 
\end{figure}


Dabei entspricht das ungefähr einer Rotation des Bildes... blbla gemäss Notizen

\begin{figure}[H]
	\center
	\includegraphics[scale=0.43]{Grafiken/entwicklung/22AngleCorrectur.jpg}
	\caption{Ergebnisse nach Filterung der Konturen nach Winkel} 
	\label{fig: Ergebnisse nach Filterung der Konturen nach Winkel} 
\end{figure}

Um diese Ergebnisse weiter zu analysieren, wurde mittels \texttt{minAreaRect()} ein für jede Kontur das Rechteck ermittelt, welches mit der kleinsten Fläche sämtliche Punkte der Kontur einschliesst \ref{fig: Rechteck mit kleinsten Fläche, welches die Kontur einschliesst}.  
\begin{figure}[H]
	\center
	\includegraphics[scale=0.43]{Grafiken/entwicklung/23ShapeAnalysis.jpg}
	\caption{Rechteck mit kleinsten Fläche, welches die Kontur einschliesst} 
	\label{fig: Rechteck mit kleinsten Fläche, welches die Kontur einschliesst} 
\end{figure}

Bei Betrachtung dieser Ergebnisse fällt auf, dass sich die Beine der Kuh insbesondere bei zwei Eigenschaften stark von der weiteren, unerwünschten Kontur unterscheidet. 

1. Während die Rechtecke, welche die Beine einschliessen langgezogen sind, ist das andere Rechteck weniger langgezogen. Das Verhältnis zwischen der Höhe und der Breite der Rechtecke (engl: "Aspect Ratio") könnte also weitere Hinweise liefern.

2. Das Verhältnis zwischen der Fläche der Kontur (der Beine) und des Rechtecks unterscheided sich zudem stark (engl: Extent). Abbildung \ref{fig :Verhältnis zwischen Konturfläche und der Fläche des Rechtecks, welches die Kontur einschliesst} verdeutlicht dies.


\begin{figure}[H]
	\center
	\includegraphics[scale=0.43]{Grafiken/entwicklung/24ExtentDemonstration.jpg}
	\caption{Verhältnis zwischen Konturfläche und der Fläche des Rechtecks, welches die Kontur einschliesst } 
	\label{fig :Verhältnis zwischen Konturfläche und der Fläche des Rechtecks, welches die Kontur einschliesst} 
\end{figure}