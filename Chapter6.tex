% ================================================================
% CHAPTER 6: Ausblick
% ================================================================


\chapter{Ausblick}
Die Resultate aus der vorliegenden Arbeit können als Grundlage für den Einsatz von Machine Learning verwendet werden.
Im Supervised Learning werden Trainingsdaten mit deren Gruppenzugehörigkeit annotiert. \citep[S. 253]{Firouzi et al. 2020}. 
Dementsprechend wird fürs Supervised Learning eine Menge von Daten benötigt, bei dem das gewünschte Ergebnis bereits bekannt ist. Algorithmen des Supervised Learning analysieren die Trainingsdaten, um eine Vorhersage über genau diese Trainingsdaten zu machen. Diese Vorhersagen können bei Abweichungen zum gewünschten Ergebnis korrigiert werden. Diese Korrekturen nutzt der Algorithmus, um die Genauigkeit der Vorhersagen zu optimieren. Im Kontext von Machine Learning werden bei einem Klassifikationsproblem die Eingabedaten einer Menge von Klassen oder Kategorien zugeordnet. \citep[S. 440]{FernandezVillan2019}.

Im Kontext der vorliegenden Arbeit können Bilder als Eingabedaten den Kategorien \flqq Geburt anstehend\frqq oder \flqq Geburt nicht anstehend\frqq zugeordnet werden. Falls ein Bild der Kateogorie \flqq Geburt anstehend\frqq zugeordnet wird, ist eine weitere Klassifikation nach \flqq Vorbereitungsphase\frqq, \flqq Öffnungsphase\frqq, \flqq Aufweitungsphase\frqq, \flqq Austreibungsphase\frqq und \flqq Nachgeburtsphase\frqq denkbar. Weiter könnte ein Bild als Eingabe Grundlage für die Risikobewertung der aktuellen Situation liefern. Dementsprechend könnte ein Bild nach \flqq einfache Geburt\frqq, \flqq normal\frqq, \flqq gefährlich\frqq und \flqq lebensbedrohlich\frqq klassifiziert werden. 

Das entwickelte System könnte genau diese Trainingsdaten liefern und jedes Bild aus den Trainingsdaten mit einem oder mehreren \flqq Tags\frqq versehen. So kann das entwickelte System als Grundlage für ein noch präziseres System dienen.

Eine Klassifikation bedeuten im vorligenden 
. Im Supervised
Learning werden in erster Linie Klassifikationsmodelle und Regressionsmodelle
angewendet. . Angewendet auf die vorliegende Arbeit könnte ein
Klassifikationsmodell unter Berücksichtigung von Geburtsmerkmalen bestimmen,
ob ein Kalb eine einfache, mittlere oder schwere Geburt haben wird.

Mit dem Einsatz von unterschiedlichen Sensoren können beispielsweise die Körpertemperatur,
der Herzschlag, verschiedene pH-Werte und Hormonprofile gemessen
werden. Der Körper des Tiers wird zum Generator von Daten. Diese werden mittels
strukturierten Verfahren für Analysen, Auswertungen, Prognosen von medizinischen
Parametern und datengestützte Profilerstellungen genutzt. 
Die Erstellung von einer elektronischen Identität eines Tieres erlaubt eine Effizenzsteigerung von
automatischen Fütterungssystemen, Melkanlagen oder Wiege-, Verlade- und Sortiereinrichtungen.
Hyperkonnektivität und Prognosen stellen durch die Schaffung eines
digitalisierten Bauernhofs neue Chancen in der Landwirtschaft dar. (Kasprowicz 
Rieger 2019).


Durch den gezielten Einsatz der beschriebenen Trends und Technologien rückt
das Tier ins Zentrum der Analyse in der Landwirtschaft. Dessen anatomischen,
physiologischen oder verhaltensspezifischen Merkmale werden untersucht. Dies
erlaubt, gezielte Massnahmen wie beispielsweise manuelle Geburtshilfe oder die
Alarmierung eines Tierarztes einzuleiten. Daraus kann ein erheblicher wirtschaftlicher
Mehrwert geschaffen werden und das Wohlergehen von Tieren kann positiv
beeinflusst werden.
Zudem können gesammelte Sensordaten durchaus für diverse Anwendungsszenarien
verwendet werden und dadurch kann der Mehrwert der entwickelten Produkte
gesteigert werden. Produkte sollen nach Möglichkeit so aufgebaut werden, dass
diese weitere Einsatzgebiete wie die Brunsterkennung, das Gesundheitsmonitoring,
oder die automatische Fütterung ermöglichen.
Zusammenfassend ist der Autor dementsprechend optimistisch, dass grosser Mehrwert
mittels Anwendung von Machine Learning für die Geburtsüberwachung von
Kälbern geschaffen werden kann.

Komplexe Systeme erstellen mittels Algorithmen
des Machine Learning Tierprofile anhand von diesen prädiktiven Merkmalen. Daraus
resultiert eine Prognose, wann sich ein Tier in der fruchtbaren Phase befindet
und dadurch der optimale Besamungszeitpunkt. (Kasprowicz  Rieger 2019).



\textbf{Der Inhalt dieses Kapitels kann im Moment als Notiz betrachtet werden. Ist noch nicht reingeschrieben.} \newline
More data regarding prepartum behavioural changes are now needed from animals of other breeds reared under different conditions. Beyond their usefulness for farmers and veterinarians, automated detection systems may also provide valuable data on pre-calving physiological changes in a large number of animals. This could become a source of better knowledge that eventually will lead to more efficient monitoring systems. Explor- ing the relationship between calving phenotypic traits and genotypic data might lead to the identification of genomic regions influenc- ing calving and eventually open the possibility for cows that calve more easily and with less perinatal mortality. \cite{Saint-Dizier2015}

Weiterentwicklungsideen: 
\begin{itemize}
	\item Machine Learning (wurde nicht gemacht, weil zu wenige Daten vorhanden) --> Hierfür könnte man den Algorithmus "Haar Cascade" zum Erkennen der Kühe/ Merkmale verwenden. --> hierfür werden viele Bilder berücksichtigt. Um dies zu erreichen eine datei positive-description und eine datei <negative-description> erstellen. die datei <positive-description> listet Bilder auf, wie oft ein Merkmal vorkommt und anschliessend die Koordinaten des Merkmals als Rechteck. Die Datei <negative-description> listet Dateien auf, in welchen das Merkmal nicht vorkommt. \citep[S. 75 ff. ]{Howse2016} und \citep[S. 139 ff. ]{Howse2016}
	\item Verbesserung der Objekterkennung anhand von Depth Cameras: Kapitel 5 von Joseph Howse et al. Dadurch könnte man nur den Vordergrund anschauen. \citep[S. 91 ff. ]{Howse2016}
	\item Feature Detection und Perspective Transforms \citep[S. 385 ff. ]{Howse2016}
\end{itemize}