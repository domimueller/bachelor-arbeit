% ================================================================
% CHAPTER 6: Ausblick
% ================================================================


\chapter{Ausblick}

\textbf{Der Inhalt dieses Kapitels kann im Moment als Notiz betrachtet werden. Ist noch nicht reingeschrieben.} \newline
More data regarding prepartum behavioural changes are now needed from animals of other breeds reared under different conditions. Beyond their usefulness for farmers and veterinarians, automated detection systems may also provide valuable data on pre-calving physiological changes in a large number of animals. This could become a source of better knowledge that eventually will lead to more efficient monitoring systems. Explor- ing the relationship between calving phenotypic traits and genotypic data might lead to the identification of genomic regions influenc- ing calving and eventually open the possibility for cows that calve more easily and with less perinatal mortality. \cite{Saint-Dizier2015}

Weiterentwicklungsideen: 
\begin{itemize}
	\item Machine Learning (wurde nicht gemacht, weil zu wenige Daten vorhanden) --> Hierfür könnte man den Algorithmus "Haar Cascade" zum Erkennen der Kühe/ Merkmale verwenden. --> hierfür werden viele Bilder berücksichtigt. Um dies zu erreichen eine datei positive-description und eine datei <negative-description> erstellen. die datei <positive-description> listet Bilder auf, wie oft ein Merkmal vorkommt und anschliessend die Koordinaten des Merkmals als Rechteck. Die Datei <negative-description> listet Dateien auf, in welchen das Merkmal nicht vorkommt. \citep[S. 75 ff. ]{Howse2016} und \citep[S. 139 ff. ]{Howse2016}
	\item Verbesserung der Objekterkennung anhand von Depth Cameras: Kapitel 5 von Joseph Howse et al. Dadurch könnte man nur den Vordergrund anschauen. \citep[S. 91 ff. ]{Howse2016}
	\item Feature Detection und Perspective Transforms \citep[S. 385 ff. ]{Howse2016}
\end{itemize}