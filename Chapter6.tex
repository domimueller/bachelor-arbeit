% ================================================================
% CHAPTER 6: Ausblick
% ================================================================


\chapter{Ausblick}
Das entwickelte System kann mit dem Einsatz von unterschiedlichen Sensoren ergänzt werden. Diese Sensoren können unterschiedliche medizinische Parameter wie Körpertemperatur, Herzschlag,  pH-Werte oder Hormonprofile messen. Der Körper des Tiers wird so zum Generator von Daten. Diese Daten werden mittels strukturierten Verfahren für Analysen, Auswertungen, Prognosen von medizinischen Parametern und datengestützte Profilerstellungen genutzt. Die Erstellung einer digitalen Identität eines Tieres erlaubt die Effizenzsteigerung von automatischen Fütterungssystemen, Melkanlagen oder Wiege-, Verlade- und Sortiereinrichtungen. Hyperkonnektivität und Prognosen stellen durch die Schaffung eines digitalisierten Bauernhofs neue Chancen in der Landwirtschaft dar \citep[S. 308 ff.]{Kasprowicz2019}. Darüber hinaus stellt auch die Entwicklung von Systemen zur \gls{Brunst}erkennung eine Möglichkeit zur Generierung von erheblichen Mehrwert dar \citep{Hirsbrunner2020}.


In Bezug auf die Geburtsanalyse können die Resultate der Bildanalyse aus der vorliegenden Arbeit als Grundlage für den Einsatz von Machine Learning verwendet werden.
Im Supervised Learning werden Trainingsdaten mit deren Gruppenzugehörigkeit annotiert \citep[S. 253]{Firouzi2020}. 
Dementsprechend wird für das Supervised Learning eine sehr grosse Menge an Daten der Grössenordnung von mindestens ${10^4}$ Datensätzen benötigt. Bei diesen Datensätzen ist das gewünschte Ergebnis bereits bekannt. Algorithmen des Supervised Learnings analysieren die Trainingsdaten, um eine Vorhersage über genau diese Trainingsdaten zu machen. Diese Vorhersagen können bei Abweichungen zum gewünschten Ergebnis korrigiert werden. Diese Korrekturen nutzt der Algorithmus, um die Genauigkeit der Vorhersagen zu optimieren. Im Kontext von Machine Learning werden bei einem Klassifikationsproblem die Eingabedaten einer Menge von Klassen oder Kategorien zugeordnet. \citep[S. 440]{FernandezVillan2019}.

Im Kontext der vorliegenden Arbeit können Bilder als Eingabedaten den Kategorien \flqq Geburt anstehend\frqq oder \flqq Geburt nicht anstehend\frqq zugeordnet werden. Falls ein Bild der Kateogorie \flqq Geburt anstehend\frqq zugeordnet wird, ist eine weitere Klassifikation nach \flqq Vorbereitungsphase\frqq, \flqq Öffnungsphase\frqq, \flqq Aufweitungsphase\frqq, \flqq Austreibungsphase\frqq und \flqq Nachgeburtsphase\frqq denkbar. Weiter könnte ein Bild als Eingabe Grundlage für die Risikobewertung der aktuellen Situation liefern. Dementsprechend könnte ein Bild nach \flqq einfache Geburt\frqq, \flqq normal\frqq, \flqq gefährlich\frqq und \flqq lebensbedrohlich\frqq klassifiziert werden. 

Das entwickelte System kann genau diese Trainingsdaten liefern und jedes Bild aus den Trainingsdaten mit einem oder mehreren \flqq Tags\frqq markieren. Es kann somit als Vorstufe zu einem auf Machine-Learning-Algorithmen basierenden System im Supervised-Learning-Modus dienen. Dieses kann anhand der annotierten Bilder trainiert werden und dann in der produktiven Phase Bilder von Kalbsgeburten automatisch klassifizieren. Wird diesem Erkennungssystem ein Benachrichtigungssystem nachgeschaltet, können Landwirte und TierärztInnen automatisch benachrichtigt werden.