% ================================================================
% CHAPTER 6: Ausblick
% ================================================================


\chapter{Ausblick}

Beim Precision Livestock Farming wird unter Anwendung von Technologien die
Optimierung von Haltungsbedingungen von Tieren adressiert. Damit soll das Wohlergehen
der Tiere und gleichzeitig die Effizienz in der Landwirtschaft gesteigert
werden. Um dies zu erreichen, werden Technologien aus den Bereichen Sensorik,
Robotik und Informationstechnologie angewendet (Kasprowicz  Rieger 2019). Mit
dem Einsatz von unterschiedlichen Sensoren können beispielsweise die Körpertemperatur,
der Herzschlag, verschiedene pH-Werte und Hormonprofile gemessen
werden. Der Körper des Tiers wird zum Generator von Daten. Diese werden mittels
strukturierten Verfahren für Analysen, Auswertungen, Prognosen von medizinischen
Parametern und datengestützte Profilerstellungen genutzt. Die Erstellung
von einer elektronischen Identität eines Tieres erlaubt eine Effizenzsteigerung von
automatischen Fütterungssystemen, Melkanlagen oder Wiege-, Verlade- und Sortiereinrichtungen.
Hyperkonnektivität und Prognosen stellen durch die Schaffung eines
digitalisierten Bauernhofs neue Chancen in der Landwirtschaft dar. (Kasprowicz 
Rieger 2019).

Wie bereits erwähnt, stehen die Konzepte des Machine Learning und des Internet
of Things (IoT) stark in Zusammenhang (Firouzi et al. 2020). Sogenannte «elektronische
Brandzeichen» ermöglichen, Tiere elektronisch zu identifizieren. Dies kann
beispielsweise durch die Injektion eines RFID-Chips oder die Anbringung eines
Bauchgürtels erreicht werden. Dabei werden Kühe mit einer Kommunikationsschnittstelle
ausgerüstet und werden dadurch Bestandteil des virtuellen Bauernhofs
Kasprowicz  Rieger (2019).


Durch den gezielten Einsatz der beschriebenen Trends und Technologien rückt
das Tier ins Zentrum der Analyse in der Landwirtschaft. Dessen anatomischen,
physiologischen oder verhaltensspezifischen Merkmale werden untersucht. Dies
erlaubt, gezielte Massnahmen wie beispielsweise manuelle Geburtshilfe oder die
Alarmierung eines Tierarztes einzuleiten. Daraus kann ein erheblicher wirtschaftlicher
Mehrwert geschaffen werden und das Wohlergehen von Tieren kann positiv
beeinflusst werden.
Zudem können gesammelte Sensordaten durchaus für diverse Anwendungsszenarien
verwendet werden und dadurch kann der Mehrwert der entwickelten Produkte
gesteigert werden. Produkte sollen nach Möglichkeit so aufgebaut werden, dass
diese weitere Einsatzgebiete wie die Brunsterkennung, das Gesundheitsmonitoring,
oder die automatische Fütterung ermöglichen.
Zusammenfassend ist der Autor dementsprechend optimistisch, dass grosser Mehrwert
mittels Anwendung von Machine Learning für die Geburtsüberwachung von
Kälbern geschaffen werden kann.

Komplexe Systeme erstellen mittels Algorithmen
des Machine Learning Tierprofile anhand von diesen prädiktiven Merkmalen. Daraus
resultiert eine Prognose, wann sich ein Tier in der fruchtbaren Phase befindet
und dadurch der optimale Besamungszeitpunkt. (Kasprowicz  Rieger 2019).

Im Supervised Learning werden einerseits Trainingsdaten
gesammelt und andererseits werden deren Gruppenzugehörigkeit durch
Experten annotiert (nachfolgend « Labeling» genannt). Auf dieser Grundlage wird
anschliessend ein Modell trainiert. Im Trainingsprozess sind die entsprechenden
korrekten Antworten aufgrund des Labeling der Datensätze bekannt. Im Rahmen
des Trainingsprozesses macht der Algorithmus iterativ Vorhersagen über die Trainingsdaten
und wird durch den Experten korrigiert. Der Lernprozess stoppt, sobald
der Algorithmus eine akzeptable Präzision der Vorhersagen erreicht. Im Supervised
Learning werden in erster Linie Klassifikationsmodelle und Regressionsmodelle
angewendet. (Firouzi et al. 2020). Angewendet auf die vorliegende Arbeit könnte ein
Klassifikationsmodell unter Berücksichtigung von Geburtsmerkmalen bestimmen,
ob ein Kalb eine einfache, mittlere oder schwere Geburt haben wird.

\textbf{Der Inhalt dieses Kapitels kann im Moment als Notiz betrachtet werden. Ist noch nicht reingeschrieben.} \newline
More data regarding prepartum behavioural changes are now needed from animals of other breeds reared under different conditions. Beyond their usefulness for farmers and veterinarians, automated detection systems may also provide valuable data on pre-calving physiological changes in a large number of animals. This could become a source of better knowledge that eventually will lead to more efficient monitoring systems. Explor- ing the relationship between calving phenotypic traits and genotypic data might lead to the identification of genomic regions influenc- ing calving and eventually open the possibility for cows that calve more easily and with less perinatal mortality. \cite{Saint-Dizier2015}

Weiterentwicklungsideen: 
\begin{itemize}
	\item Machine Learning (wurde nicht gemacht, weil zu wenige Daten vorhanden) --> Hierfür könnte man den Algorithmus "Haar Cascade" zum Erkennen der Kühe/ Merkmale verwenden. --> hierfür werden viele Bilder berücksichtigt. Um dies zu erreichen eine datei positive-description und eine datei <negative-description> erstellen. die datei <positive-description> listet Bilder auf, wie oft ein Merkmal vorkommt und anschliessend die Koordinaten des Merkmals als Rechteck. Die Datei <negative-description> listet Dateien auf, in welchen das Merkmal nicht vorkommt. \citep[S. 75 ff. ]{Howse2016} und \citep[S. 139 ff. ]{Howse2016}
	\item Verbesserung der Objekterkennung anhand von Depth Cameras: Kapitel 5 von Joseph Howse et al. Dadurch könnte man nur den Vordergrund anschauen. \citep[S. 91 ff. ]{Howse2016}
	\item Feature Detection und Perspective Transforms \citep[S. 385 ff. ]{Howse2016}
\end{itemize}