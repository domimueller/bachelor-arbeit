% ================================================================
% CHAPTER 3: Domänenanalyse: Erkenntnisse aus Interviews
% ================================================================

\section{Erkenntnisse aus den Interviews}

Ziel der Interviews ist einerseits, die Erkenntnisse aus der Literaturrecherche zu validieren und zu ergänzen. Andererseits sollen die Interviews auch Klarheit darüber bringen, wie stark die Anwesenheit oder Abwesenheit von bestimmten Eigenschaften als Hinweis für eine bevorstehende Entbindung gedeutet werden sollen. 

Jedes Merkmal wird bewertet, um zu beschreiben, wie stark dessen Anwesenheit oder Abwesenheit als Indikator einer bevorstehenden Geburt dient.  


\subsection{Kernaussagen aus Interviews}
Gemäss Interview mit Dr. med. vet. Samuel Kohler \citep{Kohler2020} sind viele der Merkmale, die für Geburtsprognosen untersucht werden, eine direkte Folge von erhöhtem Östrogenspiegel. Dies bestätigt, dass die in der Literaturrecherche identifizierten Merkmale wie beispielsweise Scheidenausfluss von einer entsprechenden Software als Geburtsanzeichen gedeutet werden dürfen (ibid.). Die alleinige Anwesenheit eines Merkmals reicht jedoch bis auf wenige Ausnahmen nicht aus, um eine Geburt mit hoher Wahrscheinlichkeit zu prognostizieren. Zudem ist es gemäss Dr. med. vet. Samuel Kohler nicht möglich, aufgrund einer wissenschaftlicher Grundlage Kombinationen von Merkmalen mit entsprechenden Schwellwerten zu nennen, welche mit ausreichender Wahrscheinlichkeit auf eine Geburt hinweisen und dementsprechend eine Benachrichtigung auslösen müssen.

Eine Geburt ist physiologisch betrachtet auch ohne viele der Merkmale wie rötliche Färbung der äusseren Geschlechtsorgane und auch ohne Scheidenausfluss möglich. Demzufolge kann die Abwesenheit der meisten Merkmale nicht verwendet werden, um eine anstehende Geburt in den nächsten Stunden mit Sicherheit auszuschliessen \citep{Kohler2020}. 

Besonders wichtig für die vorliegende Arbeit sind jedoch die prädikativen Merkmale, die immer eine Benachrichtigung auslösen müssen. Einerseits ist dies die Seitenlage mit gestreckten Beinen und andererseits die Anwesenheit der Wasser- oder Schleimblase. Dabei ist die Unterscheidung zwischen Wasserblase und Schleimblase nicht von Bedeutung. Unabhängig davon, welche Blase sichtbar ist, muss das System eine Benachrichtigung auslösen \citep{Kohler2020}.

Die Erkenntnisse aus dem Interview mit Dr. med. vet. Gaby Hirsbrunner \citep{Hirsbrunner2020} stützen die Kernaussagen aus dem Interview mit Dr. med. vet. Samuel Kohler und das gewonnene Wissen aus der Literaturrecherche. Seitenlage und Sichtbarkeit der Wasser- oder Schleimblase gehören erneut zu den wichtigsten Geburtsanzeichen, die auf Bildern erfasst werden können. Dies bestätigt die Meinung, dass jedes dieser Merkmale auch ohne die Anwesenheit von weiteren Geburtsanzeichen eine Benachrichtigung an den Landwirten auslösen soll. Zudem bestätigt Dr. med. vet. Gaby Hirsbrunner	die Meinung, dass die Abwesenheit der meisten Merkmale nicht in die Geburtsprognose einfliessen sollte.  

Die Aussagen von Dr. med. vet. Gaby Hirsbrunner verfeinern das Domänenwissen, indem die Merkmale Seitenlage für Anbdindehaltung und Freilauf getrennt bewertet werden. Gemäss \citep{Hirsbrunner2020} ist bei Anbindehaltung seitliches Liegen fast nur bei einer anstehenden Geburt zu beobachten ($Bewertung=9$). Bei der Haltung in Freilauf kann dies durchaus häufiger geschehen und ist dementsprechend auch ein schwächerer Indikator für eine anstehende Geburt ($Bewertung=4$).

Zudem sind weitere Erkenntnisse zur Erkennung von erschwerter Geburt für die vorliegende Arbeit besonders wertvoll. Blutiger Scheidenausfluss soll unbedingt als alarmierendes Zeichen interpretiert werden. Auch häufiges Trippeln kann als Symptom einer Verdrehung der Gebärmutter (Überwurf) auftreten. Sowohl blutiger Scheidenausfluss als auch häufiges Trippeln müssen daher eine Benachrichtigung auslösen \citep{Hirsbrunner2020}.

Das beschriebene System muss für einen zweckmässigen Einsatz in der Landwirtschaft und im Veterinärwesen zuverlässig, kostengünstig und einfach in der Handhabung sein \citep{Hirsbrunner2020}. Um möglichst viele visuelle Hinweise zu den Merkmalen zu erfassen, ist die Kamera hinter der Kuh zu platzieren \citep{Hirsbrunner2020, Kohler2020}. Zudem soll die Kamera einen möglichst breiten Blickwinkel abdecken, damit die Kuh nicht aus dem Sichtfeld verschwindet \cite{Muller2020}. Da laut der Meinung der Experten die Festlegung von Schwellwerten und Dimensionen während der Aufnahme nicht möglich ist \citep{Hirsbrunner2020, Kohler2020}, besteht der Benachrichtigungstext aus dem Namen des auftretenden Merkmals und einer zeitlichen Angabe.

\subsection{Codierung von Domänenwissen aus Interviews}

In Tabelle \ref{tab: Bewertung der Anwesenheit und Abwesenheit von Merkmalen} wird Domänenwissen von Prof. Dr. med. vet. Samuel Kohler und  Dr. med. vet. Gaby Hirsbrunner codiert. Dieses wurde bei Interviews per Videotelefonie erhoben. Beide befragte Personen haben die An- und Abwesenheit von Merkmalen in Bezug auf die Stärke des Hinweises auf eine bevorstehende Geburt bewertet

Die Tabelle \ref{tab: Bewertung der Anwesenheit und Abwesenheit von Merkmalen} besteht nebst der Aufzählung von Merkmalen aus den drei Spaltengruppen \flqq{}Anwesenheit\frqq{}, \flqq{}Abwesenheit\frqq{} und \flqq{}Mittelwert\frqq{} (horizontale Beschriftung). Die Gruppen \flqq{}Anwesenheit\frqq{} und \flqq{}Abwesenheit\frqq{} repräsentieren dabei die Rohdaten, welche auf der Durchführung von Interviews basieren. Die Spalten der Gruppe \flqq{}Mittelwert\frqq{} entsprechen jeweils dem gewichteten Mittelwert der Rohdaten aus den vorgelagerten Spaltengruppen.

Für die gesammelten Daten zur An- und Abwesenheit von Merkmalen sind je vier Spalten vorhanden. Eine Spalte repräsentiert jeweils die Bewertungen der Tierärzte und eine weitere Spalte entspricht der Gewichtung dieser Bewertung. In Klammer sind die Initialen des Interviewten angegeben, also \texttt{SK} für Prof. Dr. med. vet. Samuel Kohler und  \texttt{GH} für Dr. med. vet. Gaby Hirsbrunner.

Während der Durchführung der Interviews hatten die ExpertInnen die Möglichkeit, weitere Geburtsmerkmale hinzuzufügen und zu bewerten. Deshalb haben nicht beide Veterinäre sämtliche Merkmale bewertet. Falls einer davon die Anwesenheit oder Abwesenheit eines Merkmals nicht bewertet hat, ist dies mit $Bewertung=99$ codiert. Diese Bewertung wird dementsprechend nicht gewichtet, also \newline $Gewichtung=0$. 

Der gewichtete Mittelwert der Bewertung von An- und Abwesenheit dieser Merkmale dient als integraler Bestandteil des zu entwickelnden Systems. Dieses muss das gewonnene Domänenwissen einsetzen, um Kamerabilder mit Abkalbung von Kamerabildern ohne Abkalbung zu unterscheiden.



