% ================================================================
% CHAPTER 2: Methodisches Vorgehen 
% ================================================================
\chapter{Methodisches Vorgehen}

Wie auch in der Case-Arbeit nimmt in der Anfangsphase der Bachelor-Thesis die Business Analyse eine Schlüsselrolle ein. Für die Business Analyse stellen sowohl qualitative Interviews (per Videotelefonie durchgeführt) mit Fachleuten (Veterinäre, Landwirte) als auch Literaturrecherchen die wichtigsten Methoden dar. In diesem Rahmen sind zur Gewinnung von Expertenwissen die wesentlichen Merkmale eines Bildes zu identifizieren, welche für Prognose und computergetriebene Analyse des Geburtsverlaufs von Kälbern von Bedeutung sind.  Weiterhin müssen im Rahmen dieser Analyse merkmalsbezogene Muster und Schwellwerte festgelegt werden, die auf eine Geburt eines Kalbes hinweisen. Diese bestimmen, ob und zu welchem Zeitpunkt Landwirte bzw. Tierärzte über den Geburtsverlauf eines Kalbs informiert werden sollen. 

Zudem werden erworbene Kenntnisse aus der Vertiefungsrichtung Business Analyse\footnote{Weitere Informationen: \url{https://moodle.bfh.ch/course/view.php?id=20247}} aus dem Studiengang Wirtschaftsinformatik der Berner Fachhochschule angewendet.  Die im Rahmen der Business Analyse gewonnen Erkenntnisse werden mithilfe geeigneter Modelle veranschaulicht. Die Lieferobjekte sind durch die Vorgaben des Domain-Driven Designs bestimmt\footnote{Weitere Informationen: \citep{Evans2015} und \citep{Vernon2016}}. Als Ausgangspunkt und Entwurf des Domänenmodells dient das Mindmap aus Abbildung \ref{fig: Mindmap als Entwurf die für Modellierung der Domäne}

Auf Grundlage der Business Analyse ist ein System zu entwickeln, welches digitale Bilder von Kälbern verarbeitet, analysiert und von der Ausprägung von Merkmalen unter Berücksichtigung der gegebenen Schwellwerte ausgehend entsprechende automatische Benachrichtigungen auslöst. 

Tests und empirische Beobachtungen werden auf dem Bauernhof des Auftraggebers durchgeführt und das in der Case-Arbeit entwickelte System wird als Kamera verwendet. Eine geringe Anzahl an hochwertigen Kamerabildern von Kalbsgeburten bildet eine Einschränkung dar. Dies schliesst den Einsatz von Machine Learning Algorithmen im Rahmen der Bachelor-Thesis aus.

Für Entwurf und Implementierung des Systems sind bewährte Methoden des Software Engineerings anzuwenden. Dabei soll Information Hiding \citep[S. 764]{Sommerville2016}, Kapselung \citep[S. 150]{Deck2010} und Wiederverwendung von technischen Implementierungen \citep[S. 140]{Deck2010} berücksichtigt werden. Die Software ist modellgetrieben zu entwickeln und die erstellten Modelle, welche in UML-Notation erfasst sind, dienen als Dokumentation der Lösung.

Der Autor erarbeitet die Lieferergebnisse sowohl zur Implementierung als auch zur Dokumentation unter Berücksichtigung der genannten Methoden und trifft sich regelmässig mit dem Erstbetreuer, um einen Austausch zum Projektstand zu ermöglichen. Dieser berät den Autor fachlich und methodisch.

\section{Lösungsansatz}

Um die definierten Ziele zu erreichen, arbeitet der Autor mit der Programmiersprache Python (Version 3)\footnote{Weitere Informationen: \url{https://www.python.org/}} und benutzt für die Bildverarbeitung die Programmbibliothek OpenCV (Version 4)\footnote{Weitere Informationen: \url{https://opencv.org/}}. Die Auswahl von Python ermöglicht den Zugriff auf eine breite Palette von Softwarebibliotheken. Die Schnittstelle zu OpenCV ermöglicht die Durchführung von Bildanalysen und Python bietet weitere Softwarebibliotheken zur Konfiguration und zum Versand von Nachrichten über unterschiedliche Kommunikationskanäle. Der Einsatz von OpenCV bringt den Vorteil, von einer grossen Anzahl Entwickler eingesetzt zu werden. Dies hat zur Folge, dass benötigtes Wissen zur Verfügung steht. 

Diese Werkzeuge werden verwendet, um Bilder einzulesen, die identifizierten Muster zu erkennen und Merkmale auf die Erreichung der Schwellwerte zu überprüfen und anschliessend eine entsprechende, konfigurierbare Aktion auszuführen. Andere Komponenten können nach Bedarf hinzukommen.

Zusätzlich wird die Programmiersprache R (Version 3)\footnote{Weitere Informationen: \url{https://www.r-project.org/}} für die Auswertung der Interviews genutzt. Die Auswahl von R bietet den Nachteil, eine zweite Programmiersprache neben Python zu verwenden. Als Vorteile bietet R eine enge Verzahnung mir LaTex\footnote{\url{https://www.latex-project.org/}} und die effiziente Möglichkeit zur Auswertung der Interviews. Der Autor gewichtet die Vorteile des Einsatzes von R stärker und entscheidet sich deshalb für den Einsatz von diesen zwei Programmiersprachen.

\section{Zielerreichung}

Der gewählte Lösungsansatz ermöglicht eine vollständige Zielerreichung. Der Einsatz von Python und OpenCV ermöglicht die Erreichung von den Projektzielen A1, A2, A3, und A5. Das Projektziel A4 wird durch den Einsatz eines Rasperry Pi als Kamera erreicht. Darüber hinaus ermöglicht der Einsatz der ausgewählten Technologien die Entwicklung eines Systems, welches ebenfalls die Erfüllung der betrieblichen und optionalenn Ziele (B, C) ermöglicht.

Der Verzicht auf den Einsatz von Machine Learning beeinflusst die Genauigkeit der Ergebnisse, welche für die Geburtsprognose erreicht werden können. Diese Entscheidung begründet sich jedoch aufgrund des fehlenden Bildmaterials zur Geburt von Kälbern.


