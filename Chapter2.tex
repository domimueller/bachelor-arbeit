% ================================================================
% CHAPTER 2: Methodisches Vorgehen 
% ================================================================
\chapter{Methodisches Vorgehen}

Wie auch in der Case-Arbeit wird in der Anfangsphase der Bachelor-Thesis die Business Analyse eine Schlüsselrolle einnehmen. Für die Business Analyse stellen sowohl qualitative Interviews (per Videotelefonie durchgeführt) mit Fachleuten (Veterinäre, Landwirte) als auch Literaturrecherchen die wichtigsten Methoden dar.  In diesem Rahmen sind die wesentlichen Merkmale eines Bildes zu identifizieren, welche für Prognose und computergetriebene Analyse des Geburtsverlaufs von Kälbern von Bedeutung sind.  Weiterhin müssen im Rahmen dieser Analyse merkmalsbezogene Muster und Schwellwerte festgelegt werden, die auf eine Geburt eines Kalbes hinweisen. Diese bestimmen, ob und zu welchem Zeitpunkt Landwirte bzw. Tierärzte über den Geburtsverlauf eines Kalbs informiert werden sollen. 

Zudem werden erworbene Kenntnisse aus der Vertiefungsrichtung Business Analyse angewendet.  Die im Rahmen der Business Analyse gewonnen Erkenntnisse werden mithilfe geeigneter Modelle veranschaulicht. Die Lieferobjekte sind durch die Vorgaben des Domain-Driven Designs bestimmt. 

Auf Grundlage der Business Analyse ist ein System zu entwickeln, welches digitale Bilder von Kälbern verarbeitet, analysiert und von der Ausprägung von Merkmalen unter Berücksichtigung der gegebenen Schwellwerte ausgehend entsprechende automatische Benachrichtigungen auslöst. 

Für die Entwicklung des Systems sind bewährte Methoden des Software Engineerings anzuwenden. Die Software ist modellgetrieben zu entwickeln, die erstellten Modelle dienen als Dokumentation der Lösung.

Der Studierende erarbeitet die Lieferergebnisse unter Berücksichtigung der Rahmenbedingungen und trifft sich regelmässig mit dem Erstbetreuer, um einen Austausch zum Projektstand zu ermöglichen. Der Betreuer berät den Studierenden fachlich und methodisch.

\section{Lösungsansatz}

Um die definierten Ziele zu erreichen, arbeitet der Autor mit der Programmiersprache Python (Version 3) und benutzt für die Bildverarbeitung die Programmbibliothek OpenCV.

Diese Werkzeuge werden verwendet, um Bilder einzulesen, die identifizierten Muster zu erkennen und Merkmale auf die Erreichung der Schwellwerte zu überprüfen und anschliessend eine entsprechende, konfigurierbare Aktion auszuführen. Andere Komponenten können nach Bedarf hinzukommen.

Zusätzlich wir die Programmiersprache R (Version 3) für die Auswertung der Interviews genutzt.

\section{Zielerreichung}

Der gewählte Lösungsansatz ermöglicht eine (vollständige, teilweise) Zielerreichung, weil... 


