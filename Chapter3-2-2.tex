% ================================================================
% CHAPTER 3: Domänenanalyse: Erkenntnisse aus Interviews: Tabelle mit Codierung 1
% ================================================================


\newgeometry{margin=2.5cm} % Ränder kleiner	
\begin{landscape}

\begin{table}[h]
	\rowcolors{2}{maroon!10}{white!100}
	\arrayrulecolor{darkmaroon} 
	
	\setlength{\tabcolsep}{18pt} % Zellen breiter machen
	
	\begin{tabular}{  p{8.5cm}  l*{7}{l}}
 	  			
	\toprule[1pt]

	\rowcolor{maroon!30}  
	{\large \textbf{Merkmal}}&
	\rotatebox{90}{\large \textbf{{Häufigkeit }}}  & 
	\rotatebox{90}{\large \textbf{{Vorbereitungsphase (4 Tage) }}} &
	\rotatebox{90}{\large \textbf{{Vorbereitungsphase (24h) }}} & 
	\rotatebox{90}{\large \textbf{{Öffnungsphase }}} & 
	\rotatebox{90}{\large \textbf{{Aufweitungsphase}}} &
	\rotatebox{90}{\large \textbf{{Austreibungsphase }}}  &
	\rotatebox{90}{\large \textbf{{Nachgeburtsphase }}} \\
	
	\midrule[1pt]
		
Wiederkehrende Schwanzhebung & häufig & x & x &  &  &  &  \\ 
Wiederholtes Aufstehen und Abliegen & häufig &  & x &  &  &  &  \\ 
Häufiges hin-und-her-Treten (Trippeln) & häufig &  & x &  &  &  &  \\ 
Drehung des Kopfes zum Bauch hin & immer & x & x &  &  &  &  \\ 
Rote Färbung der äusseren Geschlechtsorgane & häufig &  & x & x &  &  &  \\ 
Blutiger Scheidenausfluss & häufig &  & x & x &  &  &  \\ 
Klarer Scheidenausfluss & häufig &  & x & x &  &  &  \\ 
Eingefallene Beckenbänder & immer & x & x &  &  &  &  \\ 
Euterödem & häufig & x & x &  &  &  &  \\ 
Glänzende Zitzen & häufig & x & x &  &  &  &  \\ 
Tropfende Milch & selten-häufig & x & x &  &  &  &  \\ 
Hyperplasie des Euters & immer & x &  &  &  &  &  \\ 
Schleimsekretion & häufig &  & x & x &  &  &  \\ 
Schamlippenödem & immer & x & x &  &  &  &  \\ 
Seitliches Liegen mit Abdominalkontraktion & immer &  &  & x & x & x &  \\ 
		\bottomrule

	\end{tabular}
	\caption{Zuordnung von Merkmalen zu Geburtsphasen und Bewertung der Häufigkeiten von Merkmalen (Samuel Kohler) }
	\label{tab: Zuordnung und Bewertung der Häufigkeiten von Merkmalen (Samuel Kohler)}
\end{table}

% Datentabelle von Gaby Hirsbrunner weiter unten




















\begin{table}[h]
	\rowcolors{2}{maroon!10}{white!100}
	\arrayrulecolor{darkmaroon} 
	
	\setlength{\tabcolsep}{20pt} % Zellen breiter machen
	

	\begin{tabular}{  p{8.5cm}  l*{6}{l}}
		
		\toprule[1pt]
		
		
		\rowcolor{maroon!30}  
		{\large \textbf{Merkmal}}&
		\rotatebox{90}{\large \textbf{{Häufigkeit }}}  & 
		\rotatebox{90}{\large \textbf{{Vorbereitungsphase (4 Tage) }}} &
		\rotatebox{90}{\large \textbf{{Vorbereitungsphase (24h) }}} & 
		\rotatebox{90}{\large \textbf{{Eröffnungsphase  }}} & 
		\rotatebox{90}{\large \textbf{{Austreibungsphase }}}  &
		\rotatebox{90}{\large \textbf{{Nachgeburtsphase }}}  \\

		
		\midrule[1pt]
		
Wiederkehrende Schwanzhebung & häufig &  & x & x &  &  \\ 
Wiederholtes Aufstehen und Abliegen & häufig &  &  & x &  &  \\ 
Häufiges hin-und-her-Treten (Trippeln) & selten & x & x &  &  &  \\ 
Drehung des Kopfes zum Bauch hin & häufig &  &  &  & x &  \\ 
Rote Färbung der äusseren Geschlechtsorgane & nicht zutreffend &  &  &  &  &  \\ 
Blutiger Scheidenausfluss & nicht zutreffend &  &  &  &  &  \\ 
Klarer Scheidenausfluss & immer & x &  &  &  &  \\ 
Eingefallene Beckenbänder & immer & x & x &  &  &  \\ 
Euterödem & selten & x & x &  &  &  \\ 
Glänzende Zitzen & häufig &  & x & x &  &  \\ 
Tropfende Milch & häufig &  & x & x &  &  \\ 
Hyperplasie des Euters & immer & x & x &  &  &  \\ 
Schleimsekretion & immer & x &  &  &  &  \\ 
Schamlippenödem & häufig & x & x &  &  &  \\ 
Seitliches Liegen ohne Abdominalkontraktion & häufig &  &  &  & x &  \\ 
Seitliches Liegen mit Abdominalkontraktion & immer &  &  &  & x &  \\ 
Wasserblase & immer &  &  & x &  &  \\ 
Schleimblase & immer &  &  &  & x &  \\
		\bottomrule
		
	\end{tabular}
	\caption{Zuordnung von Merkmalen zu Geburtsphasen und Bewertung der Häufigkeiten von Merkmalen (Gaby Hirsbrunner) }
	\label{tab: Zuordnung und Bewertung der Häufigkeiten von Merkmalen (Gaby Hirsbrunner)}
\end{table}




\end{landscape}
\restoregeometry % Wieder die alten Ränder



