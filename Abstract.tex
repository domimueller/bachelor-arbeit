% ================================================================
% Management Summary
% ================================================================


\chapter{Management Summary}

Auf dem Bauernhof des Arbeitgebers gebären Kühe einmal pro Jahr. Bei $15$ Kühen bedeutet dies mehr als eine Geburt pro Monat \citep{Muller2019}. Diese ist ein kritisches Ereignis für Kuh und Kalb. \gls{Dystokie} ist die Todesursache von $50$ bis $66$ Prozent der Geburten mit tödlichem Ausmass. Dies hat einerseits wirtschaftliche Konsequenzen \citep[S. 1]{Saint-Dizier2015} und ist andererseits oftmals Auslöser für nachfolgende Krankheiten \citep[S. 1]{Lange2017}.  Optimales Management und Überwachung vermindern die Eintrittswahrscheinlichkeit von Dystokie \citep[S. 1]{Lange2017}. 

Im Rahmen einer Domänenanalyse werden zur Gewinnung von Expertenwissen die wesentlichen Merkmale eines Bildes  identifiziert, welche für Prognose und computergetriebene Analyse des Geburtsverlaufs von Kälbern von Bedeutung sind. Zur Definition solcher Merkmale führt der Autor Interviews mit Fachleuten durch und betreibt Literaturrecherche. Zudem wird ein umfassendes System modelliert, welches sowohl die automatische Analyse von Kamerabildern als auch die Benachrichtigung der Stakeholder ermöglicht. Weiter erlaubt das modellierte System die Erfassung von medizinischen Daten von Kuh und Kalb.

Der Kern des modellierten Systems wird in der Programmiersprache Python mithilfe der Software-Library OpenCV zur Bildverarbeitung implementiert. Das Ergebnis ist ein flexibles und konfigurierbares System, welches die automatische Analyse von Kamerabildern bei der Geburt von Kälbern ermöglicht. Dabei wird der Fokus auf die Erkennung der Seitenlage gesetzt. Die Software ist modellgetrieben entwickelt und die erstellten Modelle, welche in UML-Notation erfasst sind, dienen als Dokumentation der Lösung.

Die Resultate der Bildanalyse können die Implementierung eines Systems mit Machine Learning unterstützen. Das entwickelte System kann jedes Bild der Trainingsdaten mit einem oder mehreren \glqq{}Tags\grqq{} markieren. So kann das entwickelte System als Grundlage für ein noch präziseres System dienen.

