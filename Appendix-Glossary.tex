% ================================================================
% CHAPTER 6: Anhang: Glossar
% ================================================================
\section{Glossar }

\begin{table}[h]
	\centering	
	\rowcolors{2}{maroon!10}{white!100}
	\arrayrulecolor{darkmaroon} 
	
	\begin{tabular}{ p{4.5cm} p{10.5cm} } 
		\toprule[1pt]
		\rowcolor{maroon!30}
		
		\textbf{Begriff} &  \textbf{kalben} \\
		\midrule
		
		Bedeutung  & ein Kalb gebären \\
		Synonyme  &  \\
		Oberbegriff  &  \\
		Unterbegriffe   & \\
		Abgrenzung, Gültigkeit  & Landwirtschaft \\		
		Eigenschaften  & \\
		Querverweise  & \\
		
		\bottomrule
		
	\end{tabular}
	\label{tab: Glossareintrag zu kalben}
	\caption{Glossareintrag zu kalben}
\end{table}


\begin{table}[h]
	\centering	
	\rowcolors{2}{maroon!10}{white!100}
	\arrayrulecolor{darkmaroon} 
	\begin{tabular}{ p{4.5cm} p{10.5cm} } 
		\toprule[1pt]
		\rowcolor{maroon!30}
		\textbf{Begriff} &  \textbf{Abkalbung}\\
		
		\midrule
		Bedeutung  & siehe "kalben" \\
		Synonyme  & kalben \\	
		Oberbegriff  &  \\
		Unterbegriffe   & \\
		Abgrenzung, Gültigkeit  & Landwirtschaft \\		
		Eigenschaften  & \\				
		Querverweise  & \\	
		\bottomrule				
		
	\end{tabular}
	\label{tab: Glossareintrag zu abkalben}
	\caption{Glossareintrag zu abkalben}
\end{table}

\begin{table}[h]
	\centering	
	\rowcolors{2}{maroon!10}{white!100}
	\arrayrulecolor{darkmaroon} 
	\begin{tabular}{ p{4.5cm} p{10.5cm} } 
		\toprule[1pt]
		\rowcolor{maroon!30}
		
		\textbf{Begriff} &  \textbf{Abdominalkontraktion} \\		
		\midrule
		
		Bedeutung  & Kontraktion  der Bauchmuskeln \cellcolor[RGB]{255, 255, 0} Ggf noch genauer abklären. geht es um eine Kontraktion der Bauchmuskeln im Sinne von Zusammenziehen des Rumpfs (sieht von aussen wie husten aus) oder geht es um Abduktion, also dem Wegführen der Extremität vom Körper. Oder sind damit Wehen gemeint? \\		
		Synonyme  & \\				
		Oberbegriff  &  \\		
		Unterbegriffe   &\\		
		Abgrenzung, Gültigkeit  & Medizin\\				
		Eigenschaften  & \\				
		Querverweise  & \\	
		\bottomrule				
		
	\end{tabular}
	\label{tab: Glossareintrag zu Abdominalkontraktion}
	\caption{Glossareintrag zu Abdominalkontraktion}
\end{table}


\begin{table}[h]
	\centering	
	\rowcolors{2}{maroon!10}{white!100}
	\arrayrulecolor{darkmaroon} 
	\begin{tabular}{ p{4.5cm} p{10.5cm} } 
		\toprule[1pt]
		\rowcolor{maroon!30}
		
		\textbf{Begriff} &  \textbf{Ödem} \\
		\midrule
				
		Bedeutung  & Ein Ödem (Wassereinlagerung im Gewebe) entsteht dann, wenn in einem Gewebe die Durchblutung erhöht (zusammen mit veränderter Strömungsmechanik des Blutes)  und/oder der Abfluss über Lymphe/Venen (oder auch die Wasserausscheidung über die Niere) gestört ist.
		Beispiel Euterödem: Mehre Tage vor dem Abkalben beginnt die Kuh mit dem „Aufeutern“, d.h. das Euter bereitet sich auf die kommende Laktation vor. Mit der beginnenden Milchproduktion muss das Eutergewebe stärker durchblutet werden. \cite{Swissgenetics} Weiter treten auch Schamlippenödeme häufig auf.\\		
		Synonyme  &  \\			
		Oberbegriff  &  \\		
		Unterbegriffe   & \\		
		Abgrenzung, Gültigkeit  & Veterinärmedizin, Landwirtschaft\\				
		Eigenschaften  & \\				
		Querverweise  & \\	
		\bottomrule				
		
	\end{tabular}
	\label{tab: Glossareintrag zu Ödem}
	\caption{Glossareintrag zu Ödem}
\end{table}

\begin{table}[h]
	\centering	
	\rowcolors{2}{maroon!10}{white!100}
	\arrayrulecolor{darkmaroon} 
	\begin{tabular}{ p{4.5cm} p{10.5cm} } 
		\toprule[1pt]
		\rowcolor{maroon!30}

		\textbf{Begriff} &  \textbf{Vorkalbeperiode} \\		
		\midrule
	
		Bedeutung  & Die Zeit, welche von 4 Tagen vor der Geburt bis zur Geburt reicht.\\		
		Synonyme  &\\				
		Oberbegriff  &  \\		
		Unterbegriffe   & \\		
		Abgrenzung, Gültigkeit  & Veterinärmedizin, Landwirtschaft\\			
		Eigenschaften  & \\				
		Querverweise  & \\	
		\bottomrule				
		
	\end{tabular}
	\label{tab: Glossareintrag zu Vorkalbeperiode}
	\caption{Glossareintrag zu Vorkalbeperiode}
\end{table}


\begin{table}[h]
	\centering	
	\rowcolors{2}{maroon!10}{white!100}
	\arrayrulecolor{darkmaroon} 
	\begin{tabular}{ p{4.5cm} p{10.5cm} } 
		\toprule[1pt]
		\rowcolor{maroon!30}
		\textbf{Begriff} &  \textbf{Dystokie} \\		
	
		Bedeutung  & Gestörter und/ oder verspäteter Geburtsverlauf, erschwerte Entbindung\\		
		Synonyme  & \\				
		Oberbegriff  &  \\		
		Unterbegriffe   & \\		
		Abgrenzung, Gültigkeit  & Medizin\\				
		Eigenschaften  & \\			
		Querverweise  & \\	
		\bottomrule				
		
	\end{tabular}
	\label{tab: Glossareintrag zu Vorkalbeperiode}
	\caption{Glossareintrag zu Vorkalbeperiode}
\end{table}
\begin{table}[h]
	\centering	
	\rowcolors{2}{maroon!10}{white!100}
	\arrayrulecolor{darkmaroon} 
	\begin{tabular}{ p{4.5cm} p{10.5cm} } 
		\toprule[1pt]
		\rowcolor{maroon!30}
		\textbf{Begriff} &  \textbf{Geburtsphase 2 nach Lange et al.}\\		
		\midrule
	
		Bedeutung  & Die zweite Geburtsphase beginnt mit der Erweiterung des Muttermundes durch die  Fruchtblase und endet mit der Austreibung des Fötus \cite{Lange2017}\\		
		Synonyme  & \\			
		Oberbegriff  &  \\		
		Unterbegriffe   & \\		
		Abgrenzung, Gültigkeit  & Medizin\\			
		Eigenschaften  & \\			
		Querverweise  &\\		
		\bottomrule			
		
	\end{tabular}
	\label{tab: Glossareintrag zu Geburtsphase 2 nach Lange et al.}
	\caption{Glossareintrag zu Geburtsphase 2 nach Lange et al.}
\end{table}


\begin{table}[h]
	\centering	
	\rowcolors{2}{maroon!10}{white!100}
	\arrayrulecolor{darkmaroon} 
	\begin{tabular}{ p{4.5cm} p{10.5cm} } 
		\toprule[1pt]
		\rowcolor{maroon!30}
		\textbf{Begriff} &  \textbf{Hyperplasie}\\		
		\midrule
	
		Bedeutung  & Vergrößerung eines Gewebes oder Organs durch vermehrte Zellteilung und eine damit verbundene außerordentliche Erhöhung der Zellanzahl\\		
		Synonyme  & \\				
		Oberbegriff  &  \\		
		Unterbegriffe   & \\		
		Abgrenzung, Gültigkeit  & Medizin\\				
		Eigenschaften  & \\			
		Querverweise  & \\
		\bottomrule					
		
	\end{tabular}
	\label{tab: Glossareintrag zu Hyperplasie}
	\caption{Glossareintrag zu Hyperplasie}
\end{table}



\begin{table}[h]
	\centering	
	\rowcolors{2}{maroon!10}{white!100}
	\arrayrulecolor{darkmaroon} 
	\begin{tabular}{ p{4.5cm} p{10.5cm} } 
		\toprule[1pt]
		\rowcolor{maroon!30}
		\textbf{Begriff} &  \textbf{Plazenta-Retention}\\
		\midrule
		Bedeutung  &\\		
		Synonyme  & \\				
		Oberbegriff  &  \\		
		Unterbegriffe   &\\		
		Abgrenzung, Gültigkeit  & Medizin\\				
		Eigenschaften  & \\			
		Querverweise  & \\	
		\bottomrule				
		
	\end{tabular}
	\label{tab: Glossareintrag zu Plazenta-Retention}
	\caption{Glossareintrag zu Plazenta-Retention}
\end{table}


\begin{table}[h]
	\centering	
	\rowcolors{2}{maroon!10}{white!100}
	\arrayrulecolor{darkmaroon} 
	\begin{tabular}{ p{4.5cm} p{10.5cm} } 
		\toprule[1pt]
		\rowcolor{maroon!30}
		\textbf{Begriff} &  \textbf{Allantochorions}\\		
		\midrule
		Bedeutung  & \\		
		Synonyme  & \\				
		Oberbegriff  &  \\		
		Unterbegriffe   & \\		
		Abgrenzung, Gültigkeit  & \\				
		Eigenschaften  & \\				
		Querverweise  & \\
		\bottomrule					
		
	\end{tabular}
	\label{tab: Glossareintrag zu Allantochorions}
	\caption{Glossareintrag zu Allantochorions}
\end{table}

\begin{table}[h]
	\centering	
	\rowcolors{2}{maroon!10}{white!100}
	\arrayrulecolor{darkmaroon} 
	\begin{tabular}{ p{4.5cm} p{10.5cm} } 
		\toprule[1pt]
		\rowcolor{maroon!30}
		\textbf{Begriff} &  \textbf{Hypothalamus-Hypophysen-Nebennierenrinden- Achse}\\		
		\midrule
		Bedeutung  & \\		
		Synonyme  & \\			
		Oberbegriff  &  \\		
		Unterbegriffe   & \\		
		Abgrenzung, Gültigkeit  & Medizin\\			
		Eigenschaften  &\\			
		Querverweise  & \\
		\bottomrule					
		
	\end{tabular}
	\label{tab: Glossareintrag zu Hypothalamus-Hypophysen-Nebennierenrinden- Achse}
	\caption{Glossareintrag zu Hypothalamus-Hypophysen-Nebennierenrinden- Achse}
\end{table}
