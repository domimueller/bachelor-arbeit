% ================================================================
% CHAPTER 3.1: Domänenanalyse: Literaturrecherche
% ================================================================
\section{Literaturrecherche}
\subsection{Grundlegende Geburtsanzeichen}

Zahlreiche Anzeichen liefern Hinweise auf eine bevorstehende Geburt eine Kalbes. Diese Anzeichen umfassen sowohl bestimmte Positionen und Bewegungsabläufe der Kuh als auch klarer oder blutiger Scheidenausfluss \citep[S. 4847]{Lange2017}.

Verdächtige Bewegungsabläufe als Anzeichen für eine bevorstehende Entbindung sind die wiederkehrende Schwanzhebung, häufiges Trippeln oder die Drehung des Kopfes zum Bauch hin \citep[S. 4847]{Lange2017}. Auch wiederholtes Aufstehen und Abliegen ist ein Geburtsanzeichen \citep[S. 352]{Saint-Dizier2015}. Das seitliche Liegen mit Abdominalkontraktion stellt eine verdächtige Position dar \citep[S. 4847]{Lange2017}. 

Weitere Geburtsanzeichen können eingefallene Beckenbänder, ein Euterödem, glänzende Zitzen oder tropfende Milch sein. Auch eine rote Färbung der  äusseren Geschlechtsorgane mit zäher Schleimspur liefert Hinweise auf eine Entbindung \citep[S. 6]{Traulsen2013}. Zudem weisen Hyperplasie des Euters, Schamlippenödem und Schleimsekretion auf eine bevorstehende Geburt hin \citep[S. 178]{Streyl2011}.

\begin{figure}[H]
	\center
	\includegraphics[scale=.45]{Grafiken/schwanzhebung.jpg}
	\caption{Schwanzhebung tritt vermehrt in den letzten 24 Stunden vor dem Kalben auf \citep[S. 4847 	f.]{Lange2017}.}
	\label{fig: Schwanzhebung}
\end{figure}

\subsection{Zeitliche und prädiktive Hinweise zu den Geburtsanzeichen}
Dabei ist zu beachten, dass einige Hinweise in erster Linie auf eine Entbindung innerhalb der nächsten vier Tage hinweisen (nachfolgend Vorkalbeperiode genannt), während andere Anzeichen auf eine Geburt innerhalb der nächsten 24 Stunden hinweisen. Ruhelosigkeit, wiederkehrende Schwanzhebung und die Drehung des Kopfes zum Bauch hin treten häufig 12 bis 6 Stunden vor der Geburt auf. Scheidenausfluss weist darauf hin, dass innerhalb der nächsten 6 Stunden die Geburt eintritt. \citep[S. 4847]{Lange2017}

Bereits durchgeführte Experimente von \citep[S. 4847]{Lange2017} anhand von stündlicher Observierung konnten das Kalben nur mit ungenügender Wahrscheinlichkeit  vorhersagen (35.1 bis 72.7\%). Demgegenüber konnte das Kalben für die nächsten 12 Stunden jedoch mit hoher Wahrscheinlichkeit ausgeschlossen werden (88.5 bis 97.1\%). Mit der Information, dass eine Kuh in den nächsten 12 Stunden nicht kalben wird, können Zeit und Ressourcen in der Überwachung optimiert werden. Daher sollen in der vorliegenden Arbeit auch Merkmale erarbeitet werden, welche darauf hinweisen, dass keine Geburt stattfindet.

Die wichtigsten Parameter zur Prognose des Kalbens innerhalb der nächsten 12 Stunden sind Beckenbänder, Zitzenfüllung,  Aufeutern und Scheiden- und Euterödeme. Diese Parameter erlauben eine genaue Vorhersage des Ausbleiben des Kalbens \citep[S. 180]{Streyl2011}.



\begin{figure}[H]
	\center
	\includegraphics[scale=.45]{Grafiken/observationTimes.png}
	\caption{Häufigkeit von Geburtsanzeichen während den letzten 24 Stunden vor Geburtsstadium 2 nach Lange et al.: Schwanzhebung (durchgezogene Linie), deutlicher Scheidenausfluss (kurze gestrichelte Linie), blutiger Scheidenausfluss (lange gestrichelte Linie) und seitliches Liegen mit Abdominalkontraktion (gepunktete Linie) \citep[S. 4851]{Lange2017}.}
	\label{fig: Häufigkeit von Geburtsanzeichen }
\end{figure}


\subsection{Weiterführende Geburtsanzeichen}
Die Geburt eines Kalbs wird durch die Hypothalamus-Hypophysen-Nebennieren-rinden-Achse des Fötus gesteuert. 72 Stunden vor der Geburt des Kalbs können diverse hormonelle Veränderungen beobachtet werden. Beispielsweise nimmt die fetale Produktion von Kortisol ungefähr zehn Tage vor der Geburt stark zu, was wiederum das Progesteron-Östradiol-Verhältnis im mütterlichen Blut beeinflusst. Auch Informationen zur Veränderung der Temperatur der äusseren Geschlechtsorgane und zum Wiederkauverhalten können für sensorbasierte Systeme einen deutlichen Mehrwert in Bezug auf die prädiktive Analyse schaffen. Ausserdem versuchen Kühe am Tag der Geburt vermehrt, sich von der Herde zu isolieren \citep[S. 349-352]{Saint-Dizier2015}. 
Da die vorliegende Arbeit aber auf die Analyse von Bildern fokussiert, werden diese sozialen Merkmale in der nachfolgenden Arbeit nicht mehr thematisiert. 
\subsection{Ablauf einer normalen Geburt}
Normalerweise befinden sich Kälber in der Vorderendlage. Das heisst, dass zuerst die Vorderbeine und der Kopf durch die Scham gepresst werden \citep{Muller2020}. Eine mögliche Fehlhaltung ist die Hinterendlage. Dabei wird das Kalb mit den Hinterbeinen voran abgekalbert. Weiter stellen gebeugte Gliedmassen, eine Kopfseitenhaltung oder eine Rückenquerlage Risiken für eine Geburt dar \citep[S. 17, 24-26]{Traulsen2013}.

\begin{figure}[H]
	\center
	\includegraphics[scale=.45]{Grafiken/vorderendlage.png}
	\caption{Veranschaulichung zur Vorderendlage eines Kalbes \citep[S. 17]{Traulsen2013}.}
	\label{fig: Schwanzhebung}
\end{figure}

Eine Geburt kann in fünf Phasen eingeteilt werden. Die Vorbereitungsphase, die Öffnungsphase, die Aufweitungsphase, die Austreibungsphase und die Nachgeburtsphase \citep[S. 6-8 ]{Traulsen2013}.

In der Vorbereitungsphase stellt sich die Kuh auf die bevorstehende Geburt ein. Deshalb treten die obengenannten Geburtsanzeichen auf (beispielsweise eingefallene Beckenbänder oder rote Färbung der äusseren Geschlechtsorgane) \citep[S. 6 ]{Traulsen2013}.  

Die Öffnungsphase erstreckt sich über 6 bis 16 Stunden. Der innere Muttermund öffnet sich und die Fruchtblasen treten in den Gebärmutterhals ein, damit dieser gedehnt wird. Es treten erste, leichte Wehen auf und die Kuh kann unruhig sein \citep[S. 7 ]{Traulsen2013}.  

\begin{figure}[H]
	\center
	\includegraphics[scale=.45]{Grafiken/oeffnungsphase.png}
	\caption{ Die Fruchtblasen (Wasser- und Schleimblase) weiten bei der Öffnungsphase den Geburtsweg \citep[S. 7 ]{Traulsen2013}.}
	\label{fig: Öffnungsphase}
\end{figure}

Die Zeit vom Blasensprung bis zum Durchtreten des Kopfes wird als Aufweitungsphase bezeichnet und dauert 1 bis 6 Stunden \citep[S. 7 ]{Traulsen2013}. 
\begin{figure}[H]
	\center
	\includegraphics[scale=.45]{Grafiken/aufweitungsphase.png}
	\caption{ Die Aufweitungsphase dauert bei Kühen im Normalfall 1 bis 3 Stunden. Bei Färsen kann dies bis zu 3 Stunden länger dauern. \citep[S. 7 ]{Traulsen2013}.}
	\label{fig: Öffnungsphase}
\end{figure} 

Im Rahmen der Austreibungsphase sollte das Kalb in nur 5 bis 15 Minuten nach dem Durchtritt des Kopfes durch die Scham geboren sein \citep[S. 8 ]{Traulsen2013}.  


\begin{figure}[H]
	\center
	\includegraphics[scale=.45]{Grafiken/austreibungsphase.png}
	\caption{ In der Austreibungsphase wird das Kalb zur Welt gebracht \citep[S. 8 ]{Traulsen2013}.}
	\label{fig: Öffnungsphase}
\end{figure}


Die Nachgeburtsphase dauert 6 bis 12 Stunden und dabei verliert die Kuh das restliche Fruchtwasser und die Nachgeburt \citep[S. 8 ]{Traulsen2013}.  
