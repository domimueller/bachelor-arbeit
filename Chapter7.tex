% ================================================================
% CHAPTER 7 : Selbstreflexion
% ================================================================


\chapter{Selbstreflexion}

Bei der Domänenanalyse konnte ich das erworbene Wissen übers Veterinärwesen mit Erlebnissen aus dem Alltag auf einem Bauernhof verknüpfen. Während der Durchführung der Bachelor-Arbeit war ich bei der Geburt von fünf Kälbern dabei und habe den Geburtsprozess beobachtet. Diese Beobachtungen stützten das erworbene fachliche Wissen.

Insgesamt stellte die technische Implementierung der Bildanalyse die grösste Herausforderung dar. Die Softwarebibliothek OpenCV war für mich neu und ich hatte nur wenig Vorkenntnisse im Bereich Bildverarbeitung. Während der Einarbeitung habe ich mir sehr viel technisches Wissen angeeignet. 

Im Verlauf des Projekts wurde mir bewusst, dass ich in der technischen Implementierung nicht sämtliche Geburtsanzeichen adressieren kann. Deshalb wurde der Fokus auf das Detektieren der Seitenlage gesetzt. Ende April wurden auf dem Bauernhof im Schwellibach zwei Kälber (Zwillinge) erwartet. Das erste Kalb kam trotz frühzeitiger Geburtsmassnahmen tot zur Welt. Da es bereits 3 Uhr morgens war, übernahm ich die weitere Beobachtung der Kuh. Meine Aufgabe war es, meinen Vater zu informieren, sobald die Kuh seitlich liegt. Nach 45 Minuten geschah dies. Während der Geburt schränkten die zähen \gls{Eihaut}\footnote{\label{glossar-eihaut}siehe Glossar} die Bewegungsfreiheit des Kalbs lebensbedrohlich ein und dieses war zur Befreiung der Atemwege dringend auf unsere Unterstützung angewiesen. Dies zeigte mir, dass das im Rahmen der Case-Arbeit entwickelte System bereits Einfluss auf die Gesundheit von Kuh und Kalb und vielleicht sogar auf Leben oder Tod der Tiere hatte. In Bezug auf die Bachelor-Arbeit verdeutlicht diese Situation, dass bereits ein System erheblichen Mehrwert bietet, welches den Landwirten über die Lage der Kuh informiert. 

Rückblickend schaue ich sehr positiv auf die vorliegende Arbeit zurück. Das Projekt war sowohl technisch als auch fachlich enorm spannend. Die Erkenntnisse motivieren mich, das entwickelte System als Grundlage für eine Weiterentwicklung mittels Machine Learning einzusetzen.