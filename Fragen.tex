% ================================================================
% CHAPTER 7 : Fragen
% ================================================================


\chapter{Fragen}

\begin{itemize}
	\item Bereits durchgeführte Experimente anhand von stündlicher Observierung konnten das Kalben für die nächsten 12 Stunden mit hoher Wahrscheinlichkeit ausschließen (88.5 bis 97.1\%), während die Vorhersage des Kalbens ungenügend war (35.1 bis 72.7) \%  \citep{Lange2017}. --> Das ist nicht gerade gut... Diese Observationen wurden nicht von Computern gemacht. Was bedeutet das für meine Arbeit? Wie kann ich diese Observation mit dem Computer besser gestalten als dies Menschen können? Wäre es technisch wohl möglich, Videoaufnahmen statt Kamerabilder zu analysieren und dementsprechend auch zB Schwanzbewegungen zu analysieren? Muss ich weitere Punkte adressieren wie zB Temperatur oder Beschleunigungssensoren oder können wir so weiterfahren?
	\item Soll ich beim Interview mit den Experten auf die Merkmale fokussieren, die ich durch Bilderkennung anschauen kann, oder soll ichs allgemein halten? --> falls ergänzungen nötig --> SaintDizier hat gute Zusammenstellung über Verhaltensänderung, und Hormonveränderung. Aber zB Analyse von Blutplasma scheint mir nicht sinnvoll.
	\item Zitieren: wenn ich zB im Artikel von SaintDizier einen Abschnitt zitiere, welcher dieser wiederum von einem anderen Autor hat, ist es ok einfach Saint DIzier zu zitieren? --> Sonst wird mein Text unlesbarer und das Abbildungsverzeichnis wird künstlich länger. (mit der Ausnahme, wenn ich in diesem Artikel weiter recherchiert habe. --> Ouellet et al. (2016).,  Pahl, Har- tung, Grothmann, Mahlkow-Nerge und Haeussermann (2014),  Burfeind, Suthar, Voigtsberger, Bonk und Heuwieser (2011). )
\end{itemize}

